\documentclass[a4paper,12pt]{article}

\usepackage[utf8]{inputenc}
\usepackage[T1]{fontenc}
\usepackage[ngerman]{babel}
\usepackage{amsmath, amssymb}
\usepackage{graphicx}
\usepackage{hyperref}


\title{Dokumentation SMP Projekt }
\author{Peter Wienken, Aaron Weis}
\date{\today}

\begin{document}

\maketitle

\tableofcontents
\newpage

\section{Audio Import}

\section{FFT Analyse}

\section{FLUX Detection}

    \subsection{Calculation of Logarithmic Magnitude Spectrum}

        The logarithmic magnitude spectrum is calculated using the formula:
        \begin{equation}
            L(k) = \log_{10}(|X(k)|)
        \end{equation}
        It transforms the linear magnitude spectrum into a logarithmic scale,
        which is more aligned with human auditory perception.

    \subsection{Calculation of Spectral Flux}
        The spectral flux is calculated as the difference between the logarithmic magnitude spectra of consecutive frames:
        \begin{equation}
            F(n) = \sum_{k=0}^{N-1} (L_n(k) - L_{n-1}(k))^2
        \end{equation}
        For each frame \( n \), the spectral flux \( F(n) \) is computed by summing the squared differences of the logarithmic magnitude spectra \( L_n(k) \) and \( L_{n-1}(k) \) across all frequency bins \( k \).
        The frames are typically stored in a 2D array where each column represents a frame and each row represents a frequency bin. The difference between consecutive frames can be represented as:

        \[
        \Delta L(k) = L_n(k) - L_{n-1}(k)
        \]
        Example of a 2D array representing frames and frequency bins:
        \[
        \mathbf{L} =
        \begin{bmatrix}
            1 & 2 & 3 & 6 \\
            1 & 2 & 3 & 6 \\
            1 & 2 & 3 & 6
        \end{bmatrix}
        \]
        The resulting difference between consecutive frames is:
        \[
        \Delta \mathbf{L} =
        \begin{bmatrix}
            2 & 3 & 6 \\
            2 & 3 & 6 \\
            2 & 3 & 6
        \end{bmatrix} 
        -
        \begin{bmatrix}
            1 & 2 & 3 \\
            1 & 2 & 3 \\
            1 & 2 & 3
        \end{bmatrix} 
        =
        \begin{bmatrix}
            1 & 1 & 3 \\
            1 & 1 & 3 \\
            1 & 1 & 3
        \end{bmatrix}
         \]
        Then, the spectral flux is computed by summing these differences across all frequency bins for each frame:
        \[
        F(n) = \sum_{k} |\Delta L(k)|
        \]

        \[ 
        \mathbf{F} =
        \begin{bmatrix}
            1+1+3 &
            1+1+3 &
            1+1+3
        \end{bmatrix}^T
        =
        \begin{bmatrix}
            5  & 5 &  5
        \end{bmatrix}^T
        \]  
         The Code to calculate the spectral flux from the differences is as follows:
         \small
         \begin{verbatim}
         flux = np.sum((calculated_FLUX[:, 1:] - calculated_FLUX[:, :-1]))
         \end{verbatim}
         \normalsize

    \subsection{Half-Wave Rectification}
        Half-wave rectification is applied to the spectral flux to retain only positive values, which correspond to increases in spectral energy. This is done by setting all negative values to zero:
        \begin{equation}
            F_{rect}(n) = \max(0, F(n))
        \end{equation}
        This step ensures that only significant increases in spectral energy are considered for further analysis.
        \\
        \\
        For example, given the spectral flux values:
        \[ \begin{bmatrix}
                -2 & 2 & -4 \\
                3  & 1 & 5\\
                -1 & 4 & -6
            \end{bmatrix}
            = 
            \begin{bmatrix}
                0 & 2 & 0 \\
                3 & 1 & 5\\
                0 & 4 & 0
            \end{bmatrix}
        \]
        And Summed up:
        \[ \begin{bmatrix}
                0+3+0 & 2+1+4 & 0+5+0
            \end{bmatrix}^T
            =
            \begin{bmatrix}
                3 & 7 & 5
            \end{bmatrix}^T
        \]

\section{Peak Detection}

\section{Fazit}
Hier steht das Fazit.

\end{document}